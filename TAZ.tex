\documentclass[a4paper,english,10pt,twoside]{article}
\usepackage[english]{babel}
\usepackage[T1]{fontenc}
\usepackage[utf8]{inputenc}
\usepackage[babel=true]{csquotes} % http://www.ctan.org/tex-archive/macros/latex/contrib/csquotes // lire le README pour l'installation
\usepackage{url}                % permet l'utilisation des urls avec \url{}
\usepackage[pdfborder=0]{hyperref}
\usepackage{lettrine}

%%%%%%%%%%%%%%%%%%%%%%%%%%%%%%%%%%%%%%%%%%%%%%
%%%%%%%%%%%%%%%%%%%%%%%%%%%%%%%%%%%%%%%%%%%%%%%%%%%%%%%%%%%%%%%%%%%%%%%%%%%%%%%%%%%%%%%%%%%%%%%%%%
\makeatletter
\def\clap#1{\hbox to 0pt{\hss #1\hss}}%
\def\ligne#1{%
\hbox to \hsize{%
\vbox{\centering #1}}}%
\def\haut#1#2#3{%
\hbox to \hsize{%
\rlap{\vtop{\raggedright #1}}%
\hss
%\clap{\vtop{\centering #2}}%
\hss
\llap{\vtop{\raggedleft #3}}}}%
\def\bas#1#2#3{%
\hbox to \hsize{%
\rlap{\vbox{\raggedright #1}}%
\hss
\clap{\vbox{\centering #2}}%
\hss
\llap{\vbox{\raggedleft #3}}}}%
\def\maketitle{%
\thispagestyle{empty}\vbox to \vsize{%
\haut{}{\@blurb}{}
\vfill
\vspace{1cm}
\begin{flushleft}
\usefont{OT1}{ptm}{m}{n}
\huge \@title
\end{flushleft}
\par
\hrule height 4pt
\par
\begin{flushright}
\usefont{OT1}{phv}{m}{n}
\Large \@author
\par
\end{flushright}
\vspace{1cm}
\vfill
\vfill
\bas{}{\@location \\ \@date}{}
}%
\cleardoublepage
}
\def\date#1{\def\@date{#1}}
\def\author#1{\def\@author{#1}}
\def\title#1{\def\@title{#1}}
\def\location#1{\def\@location{#1}}
\def\blurb#1{\def\@blurb{#1}}
%%%%%%%%%%%%%%%%%%%%%%%%%%%%%%%%%%%%%%%%%%%%%%%%%%%%%%%%%
\begin{document}
%\newcommand{\chapref}[1]{\ref{chap:#1} (page~\pageref{chap:#1})}

\title{T.A.Z - Temporary Autonomous Zone}
\author{Hakim Bey}
\location{Internet}
\date{\today}

\maketitle

%%%%%%%%%%%%%%%%%%%%%%%%%%%%%%%%%%%%%%%%%%%%%%%%%%%%%%%%%
%%%%%%%%%%%%%%%%%%%%%%%%%%%%%%%%%%%%%%%%%%%%%%%%%%%%%%%%%
\thispagestyle{empty}

\begin{center}
    \huge{T.A.Z.}\\ \Large{The Temporary Autonomous Zone\\
    By Hakim Bey}
\end{center}
\medskip

\begin{center}
    \textsc{Anti-copyright, 1985, 1991.\\}

\medskip
\Large{May be freely pirated \& quoted}
\end{center}

\medskip
The author \& publisher, however, would like to be informed at:
\begin{center}
Autonomedia\\
P. O. Box 568\\
Williamsburgh Station\\
Brooklyn, NY 11211-0568

\medskip
\url{https://bookstore.autonomedia.org}
\end{center}

\bigskip
\begin{center}
\subsection*{Book design \& typesetting:}
Dave Mandl

\subsection*{HTML version:}
Mike Morrison \\ \url{http://hermetic.com/bey/taz3.html#labelTAZ}

\subsection*{\LaTeX version (2011):}
Skhaen

\medskip
\url{https://gitorious.org/hakim-bey/hakim-bey-taz-en}
\end{center}


\newpage
\strut
\thispagestyle{empty}
\newpage

%%%%%%%%%%%%%%%%%%%%%%%%%%%%%%%%%%%%%%%%%%%%%%%%%%%%%%%%%

\tableofcontents
\setcounter{page}{4}
\newpage

%T. A. Z.
%The Temporary Autonomous Zone, Ontological Anarchy, Poetic Terrorism
%By Hakim Bey
%THE TEMPORARY AUTONOMOUS ZONE

%http://hermetic.com/bey/taz_cont.html



%%%%%%%%%%%%%%%%%%%%%%%%%%%%%%%%%%%%%%%%%%%%%%
\section{Pirate Utopias}

\enquote{\textit{… this time however I come as the victorious Dionysus, who will turn the world into a holiday… Not that I have much time… }}
\begin{flushright}Nietzsche \\(from his last \enquote{insane} letter to Cosima Wagner)\end{flushright}

\medskip
\lettrine{T}{he} sea-rovers and corsairs of the 18th century created an \enquote{information network} that spanned the globe: primitive and devoted primarily to grim business, the net nevertheless functioned admirably. Scattered throughout the net were islands, remote hideouts where ships could be watered and provisioned, booty traded for luxuries and necessities. Some of these islands supported \enquote{intentional communities,} whole mini-societies living consciously outside the law and determined to keep it up, even if only for a short but merry life.

\medskip
Some years ago I looked through a lot of secondary material on piracy hoping to find a study of these enclaves -- but it appeared as if no historian has yet found them worthy of analysis. (William Burroughs has mentioned the subject, as did the late British anarchist Larry Law -- but no systematic research has been carried out.) I retreated to primary sources and constructed my own theory, some aspects of which will be discussed in this essay. I called the settlements \enquote{Pirate Utopias}.

\medskip
Recently Bruce Sterling, one of the leading exponents of Cyberpunk science fiction, published a near-future romance based on the assumption that the decay of political systems will lead to a decentralized proliferation of experiments in living: giant worker-owned corporations, independent enclaves devoted to \enquote{data piracy}, Green-Social-Democrat enclaves, Zerowork enclaves, anarchist liberated zones, etc. The information economy which supports this diversity is called the Net; the enclaves (and the book's title) are Islands in the Net.

\medskip
The medieval Assassins founded a \enquote{State} which consisted of a network of remote mountain valleys and castles, separated by thousands of miles, strategically invulnerable to invasion, connected by the information flow of secret agents, at war with all governments, and devoted only to knowledge. Modern technology, culminating in the spy satellite, makes this kind of autonomy a romantic dream. No more pirate islands! In the future the same technology --  freed from all political control -- could make possible an entire world of autonomous zones. But for now the concept remains precisely science fiction -- pure speculation.

\medskip
Are we who live in the present doomed never to experience autonomy, never to stand for one moment on a bit of land ruled only by freedom? Are we reduced either to nostalgia for the past or nostalgia for the future? Must we wait until the entire world is freed of political control before even one of us can claim to know freedom? Logic and emotion unite to condemn such a supposition. Reason demands that one cannot struggle for what one does not know; and the heart revolts at a universe so cruel as to visit such injustices on our generation alone of humankind.

\medskip
To say that \enquote{I will not be free till all humans (or all sentient creatures) are free} is simply to cave in to a kind of nirvana-stupor, to abdicate our humanity, to define ourselves as losers.

\medskip
I believe that by extrapolating from past and future stories about \enquote{islands in the net} we may collect evidence to suggest that a certain kind of \enquote{free enclave} is not only possible in our time but also existent. All my research and speculation has crystallized around the concept of the \textsc{temporary autonomous zone} (hereafter abbreviated TAZ). Despite its synthesizing force for my own thinking, however, I don't intend the TAZ to be taken as more than an essay (\enquote{attempt}), a suggestion, almost a poetic fancy. Despite the occasional Ranterish enthusiasm of my language I am not trying to construct political dogma. In fact I have deliberately refrained from defining the TAZ -- I circle around the subject, firing off exploratory beams. In the end the TAZ is almost self-explanatory. If the phrase became current it would be understood without difficulty… understood in action.

%%%%%%%%%%%%%%%%%%%%%%%%%%%%%%%%%%%%%%%%%%%%%%
\section{Waiting for the Revolution}

\lettrine{H}{ow} is it that \enquote{the world turned upside-down} always manages to Right itself? Why does reaction always follow revolution, like seasons in Hell?

\medskip
Uprising, or the Latin form insurrection, are words used by historians to label failed revolutions -- movements which do not match the expected curve, the consensus-approved trajectory: revolution, reaction, betrayal, the founding of a stronger and even more oppressive State -- the turning of the wheel, the return of history again and again to its highest form: jackboot on the face of humanity forever.

\medskip
By failing to follow this curve, the up-rising suggests the possibility of a movement outside and beyond the Hegelian spiral of that \enquote{progress} which is secretly nothing more than a vicious circle. Surgo -- rise up, surge. Insurgo -- rise up, raise oneself up. A bootstrap operation. A goodbye to that wretched parody of the karmic round, historical revolutionary futility. The slogan \enquote{Revolution!} has mutated from tocsin to toxin, a malign pseudo-Gnostic fate-trap, a nightmare where no matter how we struggle we never escape that evil Aeon, that incubus the State, one State after another, every \enquote{heaven} ruled by yet one more evil angel.

\medskip
If History IS \enquote{Time,} as it claims to be, then the uprising is a moment that springs up and out of Time, violates the \enquote{law} of History. If the State IS History, as it claims to be, then the insurrection is the forbidden moment, an unforgivable denial of the dialectic -- shimmying up the pole and out of the smokehole, a shaman's maneuver carried out at an \enquote{impossible angle} to the universe. History says the Revolution attains \enquote{permanence,} or at least duration, while the uprising is \enquote{temporary.} In this sense an uprising is like a \enquote{peak experience} as opposed to the standard of \enquote{ordinary} consciousness and experience. Like festivals, uprisings cannot happen every day -- otherwise they would not be \enquote{nonordinary}. But such moments of intensity give shape and meaning to the entirety of a life. The shaman returns -- you can't stay up on the roof forever -- but things have changed, shifts and integrations have occurred -- a difference is made.

\medskip
You will argue that this is a counsel of despair. What of the anarchist dream, the Stateless state, the Commune, the autonomous zone with duration, a free society, a free culture? Are we to abandon that hope in return for some existentialist acte gratuit? The point is not to change consciousness but to change the world.

\medskip
I accept this as a fair criticism. I'd make two rejoinders nevertheless; first, revolution has never yet resulted in achieving this dream. The vision comes to life in the moment of uprising -- but as soon as \enquote{the Revolution} triumphs and the State returns, the dream and the ideal are already betrayed. I have not given up hope or even expectation of change -- but I distrust the word Revolution. Second, even if we replace the revolutionary approach with a concept of insurrection blossoming spontaneously into anarchist culture, our own particular historical situation is not propitious for such a vast undertaking. Absolutely nothing but a futile martyrdom could possibly result now from a head- on collision with the terminal State, the megacorporate information State, the empire of Spectacle and Simulation. Its guns are all pointed at us, while our meager weaponry finds nothing to aim at but a hysteresis, a rigid vacuity, a Spook capable of smothering every spark in an ectoplasm of information, a society of capitulation ruled by the image of the Cop and the absorbant eye of the TV screen.

\medskip
In short, we're not touting the TAZ as an exclusive end in itself, replacing all other forms of organization, tactics, and goals. We recommend it because it can provide the quality of enhancement associated with the uprising without necessarily leading to violence and martyrdom. The TAZ is like an uprising which does not engage directly with the State, a guerilla operation which liberates an area (of land, of time, of imagination) and then dissolves itself to re-form elsewhere/elsewhen, before the State can crush it. Because the State is concerned primarily with Simulation rather than substance, the TAZ can \enquote{occupy} these areas clandestinely and carry on its festal purposes for quite a while in relative peace. Perhaps certain small TAZs have lasted whole lifetimes because they went unnoticed, like hillbilly enclaves -- because they never intersected with the Spectacle, never appeared outside that real life which is invisible to the agents of Simulation.

\medskip
Babylon takes its abstractions for realities; precisely within this margin of error the TAZ can come into existence. Getting the TAZ started may involve tactics of violence and defense, but its greatest strength lies in its invisibility -- the State cannot recognize it because History has no definition of it. As soon as the TAZ is named (represented, mediated), it must vanish, it will vanish, leaving behind it an empty husk, only to spring up again somewhere else, once again invisible because undefinable in terms of the Spectacle. The TAZ is thus a perfect tactic for an era in which the State is omnipresent and all-powerful and yet simultaneously riddled with cracks and vacancies. And because the TAZ is a microcosm of that \enquote{anarchist dream} of a free culture, I can think of no better tactic by which to work toward that goal while at the same time experiencing some of its benefits here and now.

\medskip
In sum, realism demands not only that we give up waiting for \enquote{the Revolution} but also that we give up wanting it. \enquote{Uprising}, yes -- as often as possible and even at the risk of violence. The spasming of the Simulated State will be \enquote{spectacular}, but in most cases the best and most radical tactic will be to refuse to engage in spectacular violence, to withdraw from the area of simulation, to disappear.

\medskip
The TAZ is an encampment of guerilla ontologists: strike and run away. Keep moving the entire tribe, even if it's only data in the Web. The TAZ must be capable of defense; but both the \enquote{strike} and the \enquote{defense} should, if possible, evade the violence of the State, which is no longer a meaningful violence. The strike is made at structures of control, essentially at ideas; the defense is \enquote{invisibility,} a martial art, and \enquote{invulnerability} -- an \enquote{occult} art within the martial arts. The \enquote{nomadic war machine} conquers without being noticed and moves on before the map can be adjusted. As to the future -- Only the autonomous can plan autonomy, organize for it, create it. It's a bootstrap operation. The first step is somewhat akin to satori -- the realization that the TAZ begins with a simple act of realization. (Note: See Appendix C, quote by Renzo Novatore)

%%%%%%%%%%%%%%%%%%%%%%%%%%%%%%%%%%%%%%%%%%%%%%
\section{The Psychotopology of Everyday Life}

\lettrine{T}{he} concept of the TAZ arises first out of a critique of Revolution, and an appreciation of the Insurrection. The former labels the latter a failure; but for us uprising represents a far more interesting possibility, from the standard of a psychology of liberation, than all the \enquote{successful} revolutions of bourgeoisie, communists, fascists, etc.

\medskip
The second generating force behind the TAZ springs from the historical development I call \enquote{the closure of the map.} The last bit of Earth unclaimed by any nation-state was eaten up in 1899. Ours is the first century without terra incognita, without a frontier. Nationality is the highest principle of world governance -- not one speck of rock in the South Seas can be left open, not one remote valley, not even the Moon and planets. This is the apotheosis of \enquote{territorial gangsterism.} Not one square inch of Earth goes unpoliced or untaxed… in theory.

\medskip
The \enquote{map} is a political abstract grid, a gigantic con enforced by the carrot/stick conditioning of the \enquote{Expert} State, until for most of us the map becomes the territory- -no longer \enquote{Turtle Island,} but \enquote{the USA.} And yet because the map is an abstraction it cannot cover Earth with 1:1 accuracy. Within the fractal complexities of actual geography the map can see only dimensional grids. Hidden enfolded immensities escape the measuring rod. The map is not accurate; the map cannot be accurate.

\medskip
So -- Revolution is closed, but insurgency is open. For the time being we concentrate our force on temporary \enquote{power surges,} avoiding all entanglements with \enquote{permanent solutions.}

\medskip
And -- the map is closed, but the autonomous zone is open. Metaphorically it unfolds within the fractal dimensions invisible to the cartography of Control. And here we should introduce the concept of psychotopology (and -topography) as an alternative \enquote{science} to that of the State's surveying and mapmaking and \enquote{psychic imperialism.} Only psychotopography can draw 1:1 maps of reality because only the human mind provides sufficient complexity to model the real. But a 1:1 map cannot \enquote{control} its territory because it is virtually identical with its territory. It can only be used to suggest, in a sense gesture towards, certain features. We are looking for \enquote{spaces} (geographic, social, cultural, imaginal) with potential to flower as autonomous zones -- and we are looking for times in which these spaces are relatively open, either through neglect on the part of the State or because they have somehow escaped notice by the mapmakers, or for whatever reason. Psychotopology is the art of dowsing for potential TAZs.

\medskip
The closures of Revolution and of the map, however, are only the negative sources of the TAZ; much remains to be said of its positive inspirations. Reaction alone cannot provide the energy needed to \enquote{manifest} a TAZ. An uprising must be for something as well.

\begin{enumerate}
    \item{First, we can speak of a natural anthropology of the TAZ. The nuclear family is the base unit of consensus society, but not of the TAZ. (\enquote{Families! -- how I hate them! the misers of love!} -- Gide) The nuclear family, with its attendant \enquote{oedipal miseries,} appears to have been a Neolithic invention, a response to the \enquote{agricultural revolution} with its imposed scarcity and its imposed hierarchy. The Paleolithic model is at once more primal and more radical: the band. The typical hunter/gatherer nomadic or semi- nomadic band consists of about 50 people. Within larger tribal societies the band-structure is fulfilled by clans within the tribe, or by sodalities such as initiatic or secret societies, hunt or war societies, gender societies, \enquote{children's republics,} and so on. If the nuclear family is produced by scarcity (and results in miserliness), the band is produced by abundance -- and results in prodigality. The family is closed, by genetics, by the male's possession of women and children, by the hierarchic totality of agricultural/industrial society. The band is open -- not to everyone, of course, but to the affinity group, the initiates sworn to a bond of love. The band is not part of a larger hierarchy, but rather part of a horizontal pattern of custom, extended kinship, contract and alliance, spiritual affinities, etc. (American Indian society preserves certain aspects of this structure even now.)

\medskip
In our own post-Spectacular Society of Simulation many forces are working -- largely invisibly -- to phase out the nuclear family and bring back the band. Breakdowns in the structure of Work resonate in the shattered \enquote{stability} of the unit-home and unit-family. One's \enquote{band} nowadays includes friends, ex-spouses and lovers, people met at different jobs and pow-wows, affinity groups, special interest networks, mail networks, etc. The nuclear family becomes more and more obviously a trap, a cultural sinkhole, a neurotic secret implosion of split atoms -- and the obvious counter-strategy emerges spontaneously in the almost unconscious rediscovery of the more archaic and yet more post-industrial possibility of the band.}

\medskip
    \item{The TAZ as festival. Stephen Pearl Andrews once offered, as an image of anarchist society, the dinner party, in which all structure of authority dissolves in conviviality and celebration (see Appendix C). Here we might also invoke Fourier and his concept of the senses as the basis of social becoming -- \enquote{touch-rut} and \enquote{gastrosophy,} and his paean to the neglected implications of smell and taste. The ancient concepts of jubilee and saturnalia originate in an intuition that certain events lie outside the scope of \enquote{profane time,} the measuring-rod of the State and of History. These holidays literally occupied gaps in the calendar -- intercalary intervals. By the Middle Ages, nearly a third of the year was given over to holidays. Perhaps the riots against calendar reform had less to do with the \enquote{eleven lost days} than with a sense that imperial science was conspiring to close up these gaps in the calendar where the people's freedoms had accumulated -- a coup d'etat, a mapping of the year, a seizure of time itself, turning the organic cosmos into a clockwork universe. The death of the festival.

\medskip
Participants in insurrection invariably note its festive aspects, even in the midst of armed struggle, danger, and risk. The uprising is like a saturnalia which has slipped loose (or been forced to vanish) from its intercalary interval and is now at liberty to pop up anywhere or when. Freed of time and place, it nevertheless possesses a nose for the ripeness of events, and an affinity for the genius loci; the science of psychotopology indicates \enquote{flows of forces} and \enquote{spots of power} (to borrow occultist metaphors) which localize the TAZ spatio-temporally, or at least help to define its relation to moment and locale.

\medskip
The media invite us to \enquote{come celebrate the moments of your life} with the spurious unification of commodity and spectacle, the famous non-event of pure representation. In response to this obscenity we have, on the one hand, the spectrum of refusal (chronicled by the Situationists, John Zerzan, Bob Black et al.) -- and on the other hand, the emergence of a festal culture removed and even hidden from the would-be managers of our leisure. \enquote{Fight for the right to party} is in fact not a parody of the radical struggle but a new manifestation of it, appropriate to an age which offers TVs and telephones as ways to \enquote{reach out and touch} other human beings, ways to \enquote{Be There!}

\medskip
Pearl Andrews was right: the dinner party is already \enquote{the seed of the new society taking shape within the shell of the old} (IWW Preamble). The sixties-style \enquote{tribal gathering,} the forest conclave of eco-saboteurs, the idyllic Beltane of the neo-pagans, anarchist conferences, gay faery circles… Harlem rent parties of the twenties, nightclubs, banquets, old-time libertarian picnics -- we should realize that all these are already \enquote{liberated zones} of a sort, or at least potential TAZs. Whether open only to a few friends, like a dinner party, or to thousands of celebrants, like a Be-In, the party is always \enquote{open} because it is not \enquote{ordered}; it may be planned, but unless it \enquote{happens} it's a failure. The element of spontaneity is crucial.

\medskip
The essence of the party: face-to-face, a group of humans synergize their efforts to realize mutual desires, whether for good food and cheer, dance, conversation, the arts of life; perhaps even for erotic pleasure, or to create a communal artwork, or to attain the very transport of bliss --  in short, a \enquote{union of egoists} (as Stirner put it) in its simplest form -- or else, in Kropotkin's terms, a basic biological drive to \enquote{mutual aid.} (Here we should also mention Bataille's \enquote{economy of excess} and his theory of potlatch culture).}

\medskip
    \item{Vital in shaping TAZ reality is the concept of psychic nomadism (or as we jokingly call it, \enquote{rootless cosmopolitanism}). Aspects of this phenomenon have been discussed by Deleuze and Guattari in Nomadology and the War Machine, by Lyotard in Driftworks and by various authors in the \enquote{Oasis} issue of Semiotext(e). We use the term \enquote{psychic nomadism} here rather than \enquote{urban nomadism,} \enquote{nomadology,} \enquote{driftwork,} etc., simply in order to garner all these concepts into a single loose complex, to be studied in light of the coming- into-being of the TAZ. \enquote{The death of God,} in some ways a de-centering of the entire \enquote{European} project, opened a multi-perspectived post- ideological worldview able to move \enquote{rootlessly} from philosophy to tribal myth, from natural science to Taoism --  able to see for the first time through eyes like some golden insect's, each facet giving a view of an entirely other world.

\medskip
But this vision was attained at the expense of inhabiting an epoch where speed and \enquote{commodity fetishism} have created a tyrannical false unity which tends to blur all cultural diversity and individuality, so that \enquote{one place is as good as another.} This paradox creates \enquote{gypsies,} psychic travellers driven by desire or curiosity, wanderers with shallow loyalties (in fact disloyal to the \enquote{European Project} which has lost all its charm and vitality), not tied down to any particular time and place, in search of diversity and adventure… This description covers not only the X-class artists and intellectuals but also migrant laborers, refugees, the \enquote{homeless,} tourists, the RV and mobile-home culture -- also people who \enquote{travel} via the Net, but may never leave their own rooms (or those like Thoreau who \enquote{have travelled much -- in Concord}); and finally it includes \enquote{everybody,} all of us, living through our automobiles, our vacations, our TVs, books, movies, telephones, changing jobs, changing \enquote{lifestyles,} religions, diets, etc., etc.

\medskip
Psychic nomadism as a tactic, what Deleuze \& Guattari metaphorically call \enquote{the war machine,} shifts the paradox from a passive to an active and perhaps even \enquote{violent} mode. \enquote{God}'s last throes and deathbed rattles have been going on for such a long time -- in the form of Capitalism, Fascism, and Communism, for example -- that there's still a lot of \enquote{creative destruction} to be carried out by post-Bakuninist post-Nietzschean commandos or apaches (literally \enquote{enemies}) of the old Consensus. These nomads practice the razzia, they are corsairs, they are viruses; they have both need and desire for TAZs, camps of black tents under the desert stars, interzones, hidden fortified oases along secret caravan routes, \enquote{liberated} bits of jungle and bad-land, no-go areas, black markets, and underground bazaars.

\medskip
These nomads chart their courses by strange stars, which might be luminous clusters of data in cyberspace, or perhaps hallucinations. Lay down a map of the land; over that, set a map of political change; over that, a map of the Net, especially the counter-Net with its emphasis on clandestine information-flow and logistics -- and finally, over all, the 1:1 map of the creative imagination, aesthetics, values. The resultant grid comes to life, animated by unexpected eddies and surges of energy, coagulations of light, secret tunnels, surprises.}
\end{enumerate}

%%%%%%%%%%%%%%%%%%%%%%%%%%%%%%%%%%%%%%%%%%%%%%
\section{The Net and the Web}

\lettrine{T}{he} next factor contributing to the TAZ is so vast and ambiguous that it needs a section unto itself.

\medskip
We've spoken of the Net, which can be defined as the totality of all information and communication transfer. Some of these transfers are privileged and limited to various elites, which gives the Net a hierarchic aspect. Other transactions are open to all -- so the Net has a horizontal or non-hierarchic aspect as well. Military and Intelligence data are restricted, as are banking and currency information and the like. But for the most part the telephone, the postal system, public data banks, etc. are accessible to everyone and anyone. Thus within the Net there has begun to emerge a shadowy sort of counter-Net, which we will call the Web (as if the Net were a fishing-net and the Web were spider-webs woven through the interstices and broken sections of the Net). Generally we'll use the term Web to refer to the alternate horizontal open structure of info- exchange, the non-hierarchic network, and reserve the term counter-Net to indicate clandestine illegal and rebellious use of the Web, including actual data-piracy and other forms of leeching off the Net itself. Net, Web, and counter-Net are all parts of the same whole pattern-complex -- they blur into each other at innumerable points. The terms are not meant to define areas but to suggest tendencies.

\medskip
(Digression: Before you condemn the Web or counter-Net for its \enquote{parasitism,} which can never be a truly revolutionary force, ask yourself what \enquote{production} consists of in the Age of Simulation. What is the \enquote{productive class}? Perhaps you'll be forced to admit that these terms seem to have lost their meaning. In any case the answers to such questions are so complex that the TAZ tends to ignore them altogether and simply picks up what it can use. \enquote{Culture is our Nature} --  and we are the thieving magpies, or the hunter/gatherers of the world of CommTech.)

\medskip
The present forms of the unofficial Web are, one must suppose, still rather primitive: the marginal zine network, the BBS networks, pirated software, hacking, phone- phreaking, some influence in print and radio, almost none in the other big media -- no TV stations, no satellites, no fiber- optics, no cable, etc., etc. However the Net itself presents a pattern of changing/evolving relations between subjects (\enquote{users}) and objects (\enquote{data}). The nature of these relations has been exhaustively explored, from McLuhan to Virilio. It would take pages and pages to \enquote{prove} what by now \enquote{everyone knows.} Rather than rehash it all, I am interested in asking how these evolving relations suggest modes of implementation for the TAZ.

\medskip
The TAZ has a temporary but actual location in time and a temporary but actual location in space. But clearly it must also have \enquote{location} in the Web, and this location is of a different sort, not actual but virtual, not immediate but instantaneous. The Web not only provides logistical support for the TAZ, it also helps to bring it into being; crudely speaking one might say that the TAZ \enquote{exists} in information- space as well as in the \enquote{real world.} The Web can compact a great deal of time, as data, into an infinitesimal \enquote{space.} We have noted that the TAZ, because it is temporary, must necessarily lack some of the advantages of a freedom which experiences duration and a more-or-less fixed locale. But the Web can provide a kind of substitute for some of this duration and locale -- it can inform the TAZ, from its inception, with vast amounts of compacted time and space which have been \enquote{subtilized} as data.

\medskip
At this moment in the evolution of the Web, and considering our demands for the \enquote{face-to-face} and the sensual, we must consider the Web primarily as a support system, capable of carrying information from one TAZ to another, of defending the TAZ, rendering it \enquote{invisible} or giving it teeth, as the situation might demand. But more than that: If the TAZ is a nomad camp, then the Web helps provide the epics, songs, genealogies and legends of the tribe; it provides the secret caravan routes and raiding trails which make up the flowlines of tribal economy; it even contains some of the very roads they will follow, some of the very dreams they will experience as signs and portents.

\medskip
The Web does not depend for its existence on any computer technology. Word-of-mouth, mail, the marginal zine network, \enquote{phone trees,} and the like already suffice to construct an information webwork. The key is not the brand or level of tech involved, but the openness and horizontality of the structure. Nevertheless, the whole concept of the Net implies the use of computers. In the SciFi imagination the Net is headed for the condition of Cyberspace (as in Tron or Neuromancer) and the pseudo-telepathy of \enquote{virtual reality.} As a Cyberpunk fan I can't help but envision \enquote{reality hacking} playing a major role in the creation of TAZs. Like Gibson and Sterling I am assuming that the official Net will never succeed in shutting down the Web or the counter-Net -- that data-piracy, unauthorized transmissions and the free flow of information can never be frozen. (In fact, as I understand it, chaos theory predicts that any universal Control-system is impossible.)

\medskip
However, leaving aside all mere speculation about the future, we must face a very serious question about the Web and the tech it involves. The TAZ desires above all to avoid mediation, to experience its existence as immediate. The very essence of the affair is \enquote{breast-to-breast} as the sufis say, or face-to-face. But, BUT: the very essence of the Web is mediation. Machines here are our ambassadors -- the flesh is irrelevant except as a terminal, with all the sinister connotations of the term.

\medskip
The TAZ may perhaps best find its own space by wrapping its head around two seemingly contradictory attitudes toward Hi- Tech and its apotheosis the Net: (1) what we might call the Fifth Estate/Neo-Paleolithic Post-Situ Ultra-Green position, which construes itself as a luddite argument against mediation and against the Net; and (2) the Cyberpunk utopianists, futuro-libertarians, Reality Hackers and their allies who see the Net as a step forward in evolution, and who assume that any possible ill effects of mediation can be overcome -- at least, once we've liberated the means of production.

\medskip
The TAZ agrees with the hackers because it wants to come into being -- in part -- through the Net, even through the mediation of the Net. But it also agrees with the greens because it retains intense awareness of itself as body and feels only revulsion for CyberGnosis, the attempt to transcend the body through instantaneity and simulation. The TAZ tends to view the Tech/anti-Tech dichotomy as misleading, like most dichotomies, in which apparent opposites turn out to be falsifications or even hallucinations caused by semantics. This is a way of saying that the TAZ wants to live in this world, not in the idea of another world, some visionary world born of false unification (all green OR all metal) which can only be more pie in the sky by-\&-by (or as Alice put it, \enquote{Jam yesterday or jam tomorrow, but never jam today}).

\medskip
The TAZ is \enquote{utopian} in the sense that it envisions an intensification of everyday life, or as the Surrealists might have said, life's penetration by the Marvelous. But it cannot be utopian in the actual meaning of the word, nowhere, or NoPlace Place. The TAZ is somewhere. It lies at the intersection of many forces, like some pagan power- spot at the junction of mysterious ley-lines, visible to the adept in seemingly unrelated bits of terrain, landscape, flows of air, water, animals. But now the lines are not all etched in time and space. Some of them exist only \enquote{within} the Web, even though they also intersect with real times and places. Perhaps some of the lines are \enquote{non-ordinary} in the sense that no convention for quantifying them exists. These lines might better be studied in the light of chaos science than of sociology, statistics, economics, etc. The patterns of force which bring the TAZ into being have something in common with those chaotic \enquote{Strange Attractors} which exist, so to speak, between the dimensions.

\medskip
The TAZ by its very nature seizes every available means to realize itself -- it will come to life whether in a cave or an L-5 Space City -- but above all it will live, now, or as soon as possible, in however suspect or ramshackle a form, spontaneously, without regard for ideology or even anti- ideology. It will use the computer because the computer exists, but it will also use powers which are so completely unrelated to alienation or simulation that they guarantee a certain psychic paleolithism to the TAZ, a primordial-shamanic spirit which will \enquote{infect} even the Net itself (the true meaning of Cyberpunk as I read it). Because the TAZ is an intensification, a surplus, an excess, a potlatch, life spending itself in living rather than merely surviving (that snivelling shibboleth of the eighties), it cannot be defined either by Tech or anti-Tech. It contradicts itself like a true despiser of hobgoblins, because it wills itself to be, at any cost in damage to \enquote{perfection,} to the immobility of the final.

\medskip
In the Mandelbrot Set and its computer-graphic realization we watch -- in a fractal universe -- maps which are embedded and in fact hidden within maps within maps etc. to the limits of computational power. What is it for, this map which in a sense bears a 1:1 relation with a fractal dimension? What can one do with it, other than admire its psychedelic elegance?

\medskip
If we were to imagine an information map -- a cartographic projection of the Net in its entirety -- we would have to include in it the features of chaos, which have already begun to appear, for example, in the operations of complex parallel processing, telecommunications, transfers of electronic \enquote{money,} viruses, guerilla hacking and so on.

\medskip
Each of these \enquote{areas} of chaos could be represented by topographs similar to the Mandelbrot Set, such that the \enquote{peninsulas} are embedded or hidden within the map -- such that they seem to \enquote{disappear.} This \enquote{writing} -- parts of which vanish, parts of which efface themselves -- represents the very process by which the Net is already compromised, incomplete to its own view, ultimately un-Controllable. In other words, the M Set, or something like it, might prove to be useful in \enquote{plotting} (in all senses of the word) the emergence of the counterNet as a chaotic process, a \enquote{creative evolution} in Prigogine's term. If nothing else the M Set serves as a metaphor for a \enquote{mapping} of the TAZ's interface with the Net as a disappearance of information. Every \enquote{catastrophe} in the Net is a node of power for the Web, the counter-Net. The Net will be damaged by chaos, while the Web may thrive on it.

\medskip
Whether through simple data-piracy, or else by a more complex development of actual rapport with chaos, the Web- hacker, the cybernetician of the TAZ, will find ways to take advantage of perturbations, crashes, and breakdowns in the Net (ways to make information out of \enquote{entropy}). As a bricoleur, a scavenger of information shards, smuggler, blackmailer, perhaps even cyberterrorist, the TAZ-hacker will work for the evolution of clandestine fractal connections. These connections, and the different information that flows among and between them, will form \enquote{power outlets} for the coming-into-being of the TAZ itself- -as if one were to steal electricity from the energy- monopoly to light an abandoned house for squatters.

\medskip
Thus the Web, in order to produce situations conducive to the TAZ, will parasitize the Net -- but we can also conceive of this strategy as an attempt to build toward the construction of an alternative and autonomous Net, \enquote{free} and no longer parasitic, which will serve as the basis for a \enquote{new society emerging from the shell of the old.} The counter-Net and the TAZ can be considered, practically speaking, as ends in themselves -- but theoretically they can also be viewed as forms of struggle toward a different reality.

\medskip
Having said this we must still admit to some qualms about computers, some still unanswered questions, especially about the Personal Computer.

\medskip
The story of computer networks, BBSs and various other experiments in electro-democracy has so far been one of hobbyism for the most part. Many anarchists and libertarians have deep faith in the PC as a weapon of liberation and self-liberation -- but no real gains to show, no palpable liberty.

\medskip
I have little interest in some hypothetical emergent entrepreneurial class of self-employed data/word processors who will soon be able to carry on a vast cottage industry or piecemeal shitwork for various corporations and bureaucracies. Moreover it takes no ESP to foresee that this \enquote{class} will develop its underclass -- a sort of lumpen yuppetariat: housewives, for example, who will provide their families with \enquote{second incomes} by turning their own homes into electro-sweatshops, little Work-tyrannies where the \enquote{boss} is a computer network.

\medskip
Also I am not impressed by the sort of information and services proffered by contemporary \enquote{radical} networks. Somewhere -- one is told -- there exists an \enquote{information economy.} Maybe so; but the info being traded over the \enquote{alternative} BBSs seems to consist entirely of chitchat and techie-talk. Is this an economy? or merely a pastime for enthusiasts? OK, PCs have created yet another \enquote{print revolution} -- OK, marginal webworks are evolving -- OK, I can now carry on six phone conversations at once. But what difference has this made in my ordinary life?

\medskip
Frankly, I already had plenty of data to enrich my perceptions, what with books, movies, TV, theater, telephones, the U.S. Postal Service, altered states of consciousness, and so on. Do I really need a PC in order to obtain yet more such data? You offer me secret information? Well… perhaps I'm tempted -- but still I demand marvelous secrets, not just unlisted telephone numbers or the trivia of cops and politicians. Most of all I want computers to provide me with information linked to real goods -- \enquote{the good things in life,} as the IWW Preamble puts it. And here, since I'm accusing the hackers and BBSers of irritating intellectual vagueness, I must myself descend from the baroque clouds of Theory \& Critique and explain what I mean by \enquote{real goods.}

\medskip
Let's say that for both political and personal reasons I desire good food, better than I can obtain from Capitalism --  unpolluted food still blessed with strong and natural flavors. To complicate the game imagine that the food I crave is illegal -- raw milk perhaps, or the exquisite Cuban fruit mamey, which cannot be imported fresh into the U.S. because its seed is hallucinogenic (or so I'm told). I am not a farmer. Let's pretend I'm an importer of rare perfumes and aphrodisiacs, and sharpen the play by assuming most of my stock is also illegal. Or maybe I only want to trade word processing services for organic turnips, but refuse to report the transaction to the IRS (as required by law, believe it or not). Or maybe I want to meet other humans for consensual but illegal acts of mutual pleasure (this has actually been tried, but all the hard-sex BBSs have been busted -- and what use is an underground with lousy security?). In short, assume that I'm fed up with mere information, the ghost in the machine. According to you, computers should already be quite capable of facilitating my desires for food, drugs, sex, tax evasion. So what's the matter? Why isn't it happening?

\medskip
The TAZ has occurred, is occurring, and will occur with or without the computer. But for the TAZ to reach its full potential it must become less a matter of spontaneous combustion and more a matter of \enquote{islands in the Net.} The Net, or rather the counter-Net, assumes the promise of an integral aspect of the TAZ, an addition that will multiply its potential, a \enquote{quantum jump} (odd how this expression has come to mean a big leap) in complexity and significance. The TAZ must now exist within a world of pure space, the world of the senses. Liminal, even evanescent, the TAZ must combine information and desire in order to fulfill its adventure (its \enquote{happening}), in order to fill itself to the borders of its destiny, to saturate itself with its own becoming.

\medskip
Perhaps the Neo-Paleolithic School are correct when they assert that all forms of alienation and mediation must be destroyed or abandoned before our goals can be realized -- or perhaps true anarchy will be realized only in Outer Space, as some futuro-libertarians assert. But the TAZ does not concern itself very much with \enquote{was} or \enquote{will be.} The TAZ is interested in results, successful raids on consensus reality, breakthroughs into more intense and more abundant life. If the computer cannot be used in this project, then the computer will have to be overcome. My intuition however suggests that the counter-Net is already coming into being, perhaps already exists -- but I cannot prove it. I've based the theory of the TAZ in large part on this intuition. Of course the Web also involves non-computerized networks of exchange such as samizdat, the black market, etc. -- but the full potential of non-hierarchic information networking logically leads to the computer as the tool par excellence. Now I'm waiting for the hackers to prove I'm right, that my intuition is valid. Where are my turnips?

%%%%%%%%%%%%%%%%%%%%%%%%%%%%%%%%%%%%%%%%%%%%%%
\section{\enquote{Gone to Croatan}}

\lettrine{W}{e} have no desire to define the TAZ or to elaborate dogmas about how it must be created. Our contention is rather that it has been created, will be created, and is being created. Therefore it would prove more valuable and interesting to look at some TAZs past and present, and to speculate about future manifestations; by evoking a few prototypes we may be able to gauge the potential scope of the complex, and perhaps even get a glimpse of an \enquote{archetype.} Rather than attempt any sort of encyclopaedism we'll adopt a scatter-shot technique, a mosaic of glimpses, beginning quite arbitrarily with the 16th-17th centuries and the settlement of the New World.

\medskip
The opening of the \enquote{new} world was conceived from the start as an occultist operation. The magus John Dee, spiritual advisor to Elizabeth I, seems to have invented the concept of \enquote{magical imperialism} and infected an entire generation with it. Halkyut and Raleigh fell under his spell, and Raleigh used his connections with the \enquote{School of Night} -- a cabal of advanced thinkers, aristocrats, and adepts -- to further the causes of exploration, colonization and mapmaking. The Tempest was a propaganda-piece for the new ideology, and the Roanoke Colony was its first showcase experiment.

\medskip
The alchemical view of the New World associated it with materia prima or hyle, the \enquote{state of Nature,} innocence and all-possibility (\enquote{Virgin-ia}), a chaos or inchoateness which the adept would transmute into \enquote{gold,} that is, into spiritual perfection as well as material abundance. But this alchemical vision is also informed in part by an actual fascination with the inchoate, a sneaking sympathy for it, a feeling of yearning for its formless form which took the symbol of the \enquote{Indian} for its focus: \enquote{Man} in the state of nature, uncorrupted by \enquote{government.} Caliban, the Wild Man, is lodged like a virus in the very machine of Occult Imperialism; the forest/animal/humans are invested from the very start with the magic power of the marginal, despised and outcaste. On the one hand Caliban is ugly, and Nature a \enquote{howling wilderness} -- on the other, Caliban is noble and unchained, and Nature an Eden. This split in European consciousness predates the Romantic/Classical dichotomy; it's rooted in Renaissance High Magic. The discovery of America (Eldorado, the Fountain of Youth) crystallized it; and it precipitated in actual schemes for colonization.

\medskip
We were taught in elementary school that the first settlements in Roanoke failed; the colonists disappeared, leaving behind them only the cryptic message \enquote{Gone To Croatan.} Later reports of \enquote{grey-eyed Indians} were dismissed as legend. What really happened, the textbook implied, was that the Indians massacred the defenseless settlers. However, \enquote{Croatan} was not some Eldorado; it was the name of a neighboring tribe of friendly Indians. Apparently the settlement was simply moved back from the coast into the Great Dismal Swamp and absorbed into the tribe. And the grey-eyed Indians were real -- they're still there, and they still call themselves Croatans.

\medskip
So -- the very first colony in the New World chose to renounce its contract with Prospero (Dee/Raleigh/Empire) and go over to the Wild Men with Caliban. They dropped out. They became \enquote{Indians,} \enquote{went native,} opted for chaos over the appalling miseries of serfing for the plutocrats and intellectuals of London.

\medskip
As America came into being where once there had been \enquote{Turtle Island,} Croatan remained embedded in its collective psyche. Out beyond the frontier, the state of Nature (i.e. no State) still prevailed -- and within the consciousness of the settlers the option of wildness always lurked, the temptation to give up on Church, farmwork, literacy, taxes --  all the burdens of civilization -- and \enquote{go to Croatan} in some way or another. Moreover, as the Revolution in England was betrayed, first by Cromwell and then by Restoration, waves of Protestant radicals fled or were transported to the New World (which had now become a prison, a place of exile). Antinomians, Familists, rogue Quakers, Levellers, Diggers, and Ranters were now introduced to the occult shadow of wildness, and rushed to embrace it.

\medskip
Anne Hutchinson and her friends were only the best known (i.e. the most upper-class) of the Antinomians -- having had the bad luck to be caught up in Bay Colony politics -- but a much more radical wing of the movement clearly existed. The incidents Hawthorne relates in \enquote{The Maypole of Merry Mount} are thoroughly historical; apparently the extremists had decided to renounce Christianity altogether and revert to paganism. If they had succeeded in uniting with their Indian allies the result might have been an Antinomian/Celtic/Algonquin syncretic religion, a sort of 17th century North American Santeria.

\medskip
Sectarians were able to thrive better under the looser and more corrupt administrations in the Caribbean, where rival European interests had left many islands deserted or even unclaimed. Barbados and Jamaica in particular must have been settled by many extremists, and I believe that Levellerish and Ranterish influences contributed to the Buccaneer \enquote{utopia} on Tortuga. Here for the first time, thanks to Esquemelin, we can study a successful New World proto-TAZ in some depth. Fleeing from hideous \enquote{benefits} of Imperialism such as slavery, serfdom, racism and intolerance, from the tortures of impressment and the living death of the plantations, the Buccaneers adopted Indian ways, intermarried with Caribs, accepted blacks and Spaniards as equals, rejected all nationality, elected their captains democratically, and reverted to the \enquote{state of Nature.} Having declared themselves \enquote{at war with all the world,} they sailed forth to plunder under mutual contracts called \enquote{Articles} which were so egalitarian that every member received a full share and the Captain usually only 1 1/4 or 1 1/2 shares. Flogging and punishments were forbidden -- quarrels were settled by vote or by the code duello.

\medskip
It is simply wrong to brand the pirates as mere sea-going highwaymen or even proto-capitalists, as some historians have done. In a sense they were \enquote{social bandits,} although their base communities were not traditional peasant societies but \enquote{utopias} created almost ex nihilo in terra incognita, enclaves of total liberty occupying empty spaces on the map. After the fall of Tortuga, the Buccaneer ideal remained alive all through the \enquote{Golden Age} of Piracy (ca. 1660-1720), and resulted in land-settlements in Belize, for example, which was founded by Buccaneers. Then, as the scene shifted to Madagascar -- an island still unclaimed by any imperial power and ruled only by a patchwork of native kings (chiefs) eager for pirate allies -- the Pirate Utopia reached its highest form.

\medskip
Defoe's account of Captain Mission and the founding of Libertatia may be, as some historians claim, a literary hoax meant to propagandize for radical Whig theory -- but it was embedded in The General History of the Pyrates (1724-28), most of which is still accepted as true and accurate. Moreover the story of Capt. Mission was not criticized when the book appeared and many old Madagascar hands still survived. They seem to have believed it, no doubt because they had experienced pirate enclaves very much like Libertatia. Once again, rescued slaves, natives, and even traditional enemies such as the Portuguese were all invited to join as equals. (Liberating slave ships was a major preoccupation.) Land was held in common, representatives elected for short terms, booty shared; doctrines of liberty were preached far more radical than even those of Common Sense.

\medskip
Libertatia hoped to endure, and Mission died in its defense. But most of the pirate utopias were meant to be temporary; in fact the corsairs' true \enquote{republics} were their ships, which sailed under Articles. The shore enclaves usually had no law at all. The last classic example, Nassau in the Bahamas, a beachfront resort of shacks and tents devoted to wine, women (and probably boys too, to judge by Birge's Sodomy and Piracy), song (the pirates were inordinately fond of music and used to hire on bands for entire cruises), and wretched excess, vanished overnight when the British fleet appeared in the Bay. Blackbeard and \enquote{Calico Jack} Rackham and his crew of pirate women moved on to wilder shores and nastier fates, while others meekly accepted the Pardon and reformed. But the Buccaneer tradition lasted, both in Madagascar where the mixed-blood children of the pirates began to carve out kingdoms of their own, and in the Caribbean, where escaped slaves as well as mixed black/white/red groups were able to thrive in the mountains and backlands as \enquote{Maroons.} The Maroon community in Jamaica still retained a degree of autonomy and many of the old folkways when Zora Neale Hurston visited there in the 1920's (see Tell My Horse). The Maroons of Suriname still practice African \enquote{paganism.}

\medskip
Throughout the 18th century, North America also produced a number of drop-out \enquote{tri-racial isolate communities.} (This clinical-sounding term was invented by the Eugenics Movement, which produced the first scientific studies of these communities. Unfortunately the \enquote{science} merely served as an excuse for hatred of racial \enquote{mongrels} and the poor, and the \enquote{solution to the problem} was usually forced sterilization.) The nuclei invariably consisted of runaway slaves and serfs, \enquote{criminals} (i.e. the very poor), \enquote{prostitutes} (i.e. white women who married non-whites), and members of various native tribes. In some cases, such as the Seminole and Cherokee, the traditional tribal structure absorbed the newcomers; in other cases, new tribes were formed. Thus we have the Maroons of the Great Dismal Swamp, who persisted through the 18th and 19th centuries, adopting runaway slaves, functioning as a way station on the Underground Railway, and serving as a religious and ideological center for slave rebellions. The religion was HooDoo, a mixture of African, native, and Christian elements, and according to the historian H. Leaming-Bey the elders of the faith and the leaders of the Great Dismal Maroons were known as \enquote{the Seven Finger High Glister.}

\medskip
The Ramapaughs of northern New Jersey (incorrectly known as the \enquote{Jackson Whites}) present another romantic and archetypal genealogy: freed slaves of the Dutch poltroons, various Delaware and Algonquin clans, the usual \enquote{prostitutes,} the \enquote{Hessians} (a catch-phrase for lost British mercenaries, drop-out Loyalists, etc.), and local bands of social bandits such as Claudius Smith's.

\medskip
An African-Islamic origin is claimed by some of the groups, such as the Moors of Delaware and the Ben Ishmaels, who migrated from Kentucky to Ohio in the mid-18th century. The Ishmaels practiced polygamy, never drank alcohol, made their living as minstrels, intermarried with Indians and adopted their customs, and were so devoted to nomadism that they built their houses on wheels. Their annual migration triangulated on frontier towns with names like Mecca and Medina. In the 19th century some of them espoused anarchist ideals, and they were targeted by the Eugenicists for a particularly vicious pogrom of salvation-by-extermination. Some of the earliest Eugenics laws were passed in their honor. As a tribe they \enquote{disappeared} in the 1920's, but probably swelled the ranks of early \enquote{Black Islamic} sects such as the Moorish Science Temple. I myself grew up on legends of the \enquote{Kallikaks} of the nearby New Jersey Pine Barrens (and of course on Lovecraft, a rabid racist who was fascinated by the isolate communities). The legends turned out to be folk-memories of the slanders of the Eugenicists, whose U.S. headquarters were in Vineland, NJ, and who undertook the usual \enquote{reforms} against \enquote{miscegenation} and \enquote{feeblemindedness} in the Barrens (including the publication of photographs of the Kallikaks, crudely and obviously retouched to make them look like monsters of misbreeding).

\medskip
The \enquote{isolate communities} -- at least, those which have retained their identity into the 20th century -- consistently refuse to be absorbed into either mainstream culture or the black \enquote{subculture} into which modern sociologists prefer to categorize them. In the 1970's, inspired by the Native American renaissance, a number of groups -- including the Moors and the Ramapaughs -- applied to the B.I.A. for recognition as Indian tribes. They received support from native activists but were refused official status. If they'd won, after all, it might have set a dangerous precedent for drop-outs of all sorts, from \enquote{white Peyotists} and hippies to black nationalists, aryans, anarchists and libertarians --  a \enquote{reservation} for anyone and everyone! The \enquote{European Project} cannot recognize the existence of the Wild Man --  green chaos is still too much of a threat to the imperial dream of order.

\medskip
Essentially the Moors and Ramapaughs rejected the \enquote{diachronic} or historical explanation of their origins in favor of a \enquote{synchronic} self-identity based on a \enquote{myth} of Indian adoption. Or to put it another way, they named themselves \enquote{Indians.} If everyone who wished \enquote{to be an Indian} could accomplish this by an act of self- naming, imagine what a departure to Croatan would take place. That old occult shadow still haunts the remnants of our forests (which, by the way, have greatly increased in the Northeast since the 18-19th century as vast tracts of farmland return to scrub. Thoreau on his deathbed dreamed of the return of \enquote{… Indians… forests… }: the return of the repressed).

\medskip
The Moors and Ramapaughs of course have good materialist reasons to think of themselves as Indians -- after all, they have Indian ancestors -- but if we view their self-naming in \enquote{mythic} as well as historical terms we'll learn more of relevance to our quest for the TAZ. Within tribal societies there exist what some anthropologists call mannenbunden: totemic societies devoted to an identity with \enquote{Nature} in the act of shapeshifting, of becoming the totem-animal (werewolves, jaguar shamans, leopard men, cat-witches, etc.). In the context of an entire colonial society (as Taussig points out in Shamanism, Colonialism and the Wild Man) the shapeshifting power is seen as inhering in the native culture as a whole --  thus the most repressed sector of the society acquires a paradoxical power through the myth of its occult knowledge, which is feared and desired by the colonist. Of course the natives really do have certain occult knowledge; but in response to Imperial perception of native culture as a kind of \enquote{spiritual wild(er)ness,} the natives come to see themselves more and more consciously in that role. Even as they are marginalized, the Margin takes on an aura of magic. Before the whiteman, they were simply tribes of people -- now, they are \enquote{guardians of Nature,} inhabitants of the \enquote{state of Nature.} Finally the colonist himself is seduced by this \enquote{myth.} Whenever an American wants to drop out or back into Nature, invariably he \enquote{becomes an Indian.} The Massachusetts radical democrats (spiritual descendents of the radical Protestants) who organized the Tea Party, and who literally believed that governments could be abolished (the whole Berkshire region declared itself in a \enquote{state of Nature}!), disguised themselves as \enquote{Mohawks.} Thus the colonists, who suddenly saw themselves marginalized vis-·- vis the motherland, adopted the role of the marginalized natives, thereby (in a sense) seeking to participate in their occult power, their mythic radiance. From the Mountain Men to the Boy Scouts, the dream of \enquote{becoming an Indian} flows beneath myriad strands of American history, culture and consciousness.

\medskip
The sexual imagery connected to \enquote{tri-racial} groups also bears out this hypothesis. \enquote{Natives} of course are always immoral, but racial renegades and drop-outs must be downright polymorphous-perverse. The Buccaneers were buggers, the Maroons and Mountain Men were miscegenists, the \enquote{Jukes and Kallikaks} indulged in fornication and incest (leading to mutations such as polydactyly), the children ran around naked and masturbated openly, etc., etc. Reverting to a \enquote{state of Nature} paradoxically seems to allow for the practice of every \enquote{unnatural} act; or so it would appear if we believe the Puritans and Eugenicists. And since many people in repressed moralistic racist societies secretly desire exactly these licentious acts, they project them outwards onto the marginalized, and thereby convince themselves that they themselves remain civilized and pure. And in fact some marginalized communities do really reject consensus morality -- the pirates certainly did! -- and no doubt actually act out some of civilization's repressed desires. (Wouldn't you?) Becoming \enquote{wild} is always an erotic act, an act of nakedness.

\medskip
Before leaving the subject of the \enquote{tri-racial isolates,} I'd like to recall Nietzsche's enthusiasm for \enquote{race mixing.} Impressed by the vigor and beauty of hybrid cultures, he offered miscegenation not only as a solution to the problem of race but also as the principle for a new humanity freed of ethnic and national chauvinism -- a precursor to the \enquote{psychic nomad,} perhaps. Nietzsche's dream still seems as remote now as it did to him. Chauvinism still rules OK. Mixed cultures remain submerged. But the autonomous zones of the Buccaneers and Maroons, Ishmaels and Moors, Ramapaughs and \enquote{Kallikaks} remain, or their stories remain, as indications of what Nietzsche might have called \enquote{the Will to Power as Disappearance.} We must return to this theme.

%%%%%%%%%%%%%%%%%%%%%%%%%%%%%%%%%%%%%%%%%%%%%%
\section{Music as an Organizational Principle}

\lettrine{M}{eanwhile}, however, we turn to the history of classical anarchism in the light of the TAZ concept.

Before the \enquote{closure of the map,} a good deal of anti- authoritarian energy went into \enquote{escapist} communes such as Modern Times, the various Phalansteries, and so on. Interestingly, some of them were not intended to last \enquote{forever,} but only as long as the project proved fulfilling. By Socialist/Utopian standards these experiments were \enquote{failures,} and therefore we know little about them.

\medskip
When escape beyond the frontier proved impossible, the era of revolutionary urban Communes began in Europe. The Communes of Paris, Lyons and Marseilles did not survive long enough to take on any characteristics of permanence, and one wonders if they were meant to. From our point of view the chief matter of fascination is the spirit of the Communes. During and after these years anarchists took up the practice of revolutionary nomadism, drifting from uprising to uprising, looking to keep alive in themselves the intensity of spirit they experienced in the moment of insurrection. In fact, certain anarchists of the Stirnerite/Nietzschean strain came to look on this activity as an end in itself, a way of always occupying an autonomous zone, the interzone which opens up in the midst or wake of war and revolution (cf. Pynchon's \enquote{zone} in Gravity's Rainbow). They declared that if any socialist revolution succeeded, they'd be the first to turn against it. Short of universal anarchy they had no intention of ever stopping. In Russia in 1917 they greeted the free Soviets with joy: this was their goal. But as soon as the Bolsheviks betrayed the Revolution, the individualist anarchists were the first to go back on the warpath. After Kronstadt, of course, all anarchists condemned the \enquote{Soviet Union} (a contradiction in terms) and moved on in search of new insurrections.

\medskip
Makhno's Ukraine and anarchist Spain were meant to have duration, and despite the exigencies of continual war both succeeded to a certain extent: not that they lasted a \enquote{long time,} but they were successfully organized and could have persisted if not for outside aggression. Therefore, from among the experiments of the inter-War period I'll concentrate instead on the madcap Republic of Fiume, which is much less well known, and was not meant to endure. Gabriele D'Annunzio, Decadent poet, artist, musician, aesthete, womanizer, pioneer daredevil aeronautist, black magician, genius and cad, emerged from World War I as a hero with a small army at his beck and command: the \enquote{Arditi.} At a loss for adventure, he decided to capture the city of Fiume from Yugoslavia and give it to Italy. After a necromantic ceremony with his mistress in a cemetery in Venice he set out to conquer Fiume, and succeeded without any trouble to speak of. But Italy turned down his generous offer; the Prime Minister called him a fool.

\medskip
In a huff, D'Annunzio decided to declare independence and see how long he could get away with it. He and one of his anarchist friends wrote the Constitution, which declared music to be the central principle of the State. The Navy (made up of deserters and Milanese anarchist maritime unionists) named themselves the Uscochi, after the long- vanished pirates who once lived on local offshore islands and preyed on Venetian and Ottoman shipping. The modern Uscochi succeeded in some wild coups: several rich Italian merchant vessels suddenly gave the Republic a future: money in the coffers! Artists, bohemians, adventurers, anarchists (D'Annunzio corresponded with Malatesta), fugitives and Stateless refugees, homosexuals, military dandies (the uniform was black with pirate skull-\&-crossbones -- later stolen by the SS), and crank reformers of every stripe (including Buddhists, Theosophists and Vedantists) began to show up at Fiume in droves. The party never stopped. Every morning D'Annunzio read poetry and manifestos from his balcony; every evening a concert, then fireworks. This made up the entire activity of the government. Eighteen months later, when the wine and money had run out and the Italian fleet finally showed up and lobbed a few shells at the Municipal Palace, no one had the energy to resist.

\medskip
D'Annunzio, like many Italian anarchists, later veered toward fascism -- in fact, Mussolini (the ex-Syndicalist) himself seduced the poet along that route. By the time D'Annunzio realized his error it was too late: he was too old and sick. But Il Duce had him killed anyway -- pushed off a balcony -- and turned him into a \enquote{martyr.} As for Fiume, though it lacked the seriousness of the free Ukraine or Barcelona, it can probably teach us more about certain aspects of our quest. It was in some ways the last of the pirate utopias (or the only modern example) -- in other ways, perhaps, it was very nearly the first modern TAZ.

\medskip
I believe that if we compare Fiume with the Paris uprising of 1968 (also the Italian urban insurrections of the early seventies), as well as with the American countercultural communes and their anarcho-New Left influences, we should notice certain similarities, such as: -- the importance of aesthetic theory (cf. the Situationists) -- also, what might be called \enquote{pirate economics,} living high off the surplus of social overproduction -- even the popularity of colorful military uniforms -- and the concept of music as revolutionary social change -- and finally their shared air of impermanence, of being ready to move on, shape-shift, re- locate to other universities, mountaintops, ghettos, factories, safe houses, abandoned farms -- or even other planes of reality. No one was trying to impose yet another Revolutionary Dictatorship, either at Fiume, Paris, or Millbrook. Either the world would change, or it wouldn't. Meanwhile keep on the move and live intensely.

\medskip
The Munich Soviet (or \enquote{Council Republic}) of 1919 exhibited certain features of the TAZ, even though -- like most revolutions -- its stated goals were not exactly \enquote{temporary.} Gustav Landauer's participation as Minister of Culture along with Silvio Gesell as Minister of Economics and other anti- authoritarian and extreme libertarian socialists such as the poet/playwrights Erich Mªhsam and Ernst Toller, and Ret Marut (the novelist B. Traven), gave the Soviet a distinct anarchist flavor. Landauer, who had spent years of isolation working on his grand synthesis of Nietzsche, Proudhon, Kropotkin, Stirner, Meister Eckhardt, the radical mystics, and the Romantic volk-philosophers, knew from the start that the Soviet was doomed; he hoped only that it would last long enough to be understood. Kurt Eisner, the martyred founder of the Soviet, believed quite literally that poets and poetry should form the basis of the revolution. Plans were launched to devote a large piece of Bavaria to an experiment in anarcho-socialist economy and community. Landauer drew up proposals for a Free School system and a People's Theater. Support for the Soviet was more or less confined to the poorest working-class and bohemian neighborhoods of Munich, and to groups like the Wandervogel (the neo-Romantic youth movement), Jewish radicals (like Buber), the Expressionists, and other marginals. Thus historians dismiss it as the \enquote{Coffeehouse Republic} and belittle its significance in comparison with Marxist and Spartacist participation in Germany's post-War revolution(s). Outmaneuvered by the Communists and eventually murdered by soldiers under the influence of the occult/fascist Thule Society, Landauer deserves to be remembered as a saint. Yet even anarchists nowadays tend to misunderstand and condemn him for \enquote{selling out} to a \enquote{socialist government.} If the Soviet had lasted even a year, we would weep at the mention of its beauty -- but before even the first flowers of that Spring had wilted, the geist and the spirit of poetry were crushed, and we have forgotten. Imagine what it must have been to breathe the air of a city in which the Minister of Culture has just predicted that schoolchildren will soon be memorizing the works of Walt Whitman. Ah for a time machine… 

%%%%%%%%%%%%%%%%%%%%%%%%%%%%%%%%%%%%%%%%%%%%%%
\section{The Will to Power as Disappearance}

\lettrine{F}{oucault}, Baudrillard, et Al. have discussed various modes of \enquote{disappearance} at great length. Here I wish to suggest that the TAZ is in some sense a tactic of disappearance. When the Theorists speak of the disappearance of the Social they mean in part the impossibility of the \enquote{Social Revolution,} and in part the impossibility of \enquote{the State} -- the abyss of power, the end of the discourse of power. The anarchist question in this case should then be: Why bother to confront a \enquote{power} which has lost all meaning and become sheer Simulation? Such confrontations will only result in dangerous and ugly spasms of violence by the emptyheaded shit-for-brains who've inherited the keys to all the armories and prisons. (Perhaps this is a crude american misunderstanding of sublime and subtle Franco-Germanic Theory. If so, fine; whoever said understanding was needed to make use of an idea?)

\medskip
As I read it, disappearance seems to be a very logical radical option for our time, not at all a disaster or death for the radical project. Unlike the morbid deathfreak nihilistic interpretation of Theory, mine intends to mine it for useful strategies in the always-ongoing \enquote{revolution of everyday life}: the struggle that cannot cease even with the last failure of political or social revolution because nothing except the end of the world can bring an end to everyday life, nor to our aspirations for the good things, for the Marvelous. And as Nietzsche said, if the world could come to an end, logically it would have done so; it has not, so it does not. And so, as one of the sufis said, no matter how many draughts of forbidden wine we drink, we will carry this raging thirst into eternity.

\medskip
Zerzan and Black have independently noted certain \enquote{elements of Refusal} (Zerzan's term) which perhaps can be seen as somehow symptomatic of a radical culture of disappearance, partly unconscious but partly conscious, which influences far more people than any leftist or anarchist idea. These gestures are made against institutions, and in that sense are \enquote{negative} -- but each negative gesture also suggests a \enquote{positive} tactic to replace rather than merely refuse the despised institution.

\medskip
For example, the negative gesture against schooling is \enquote{voluntary illiteracy.} Since I do not share the liberal worship of literacy for the sake of social ameliorization, I cannot quite share the gasps of dismay heard everywhere at this phenomenon: I sympathize with children who refuse books along with the garbage in the books. There are however positive alternatives which make use of the same energy of disappearance. Home-schooling and craft-apprenticeship, like truancy, result in an absence from the prison of school. Hacking is another form of \enquote{education} with certain features of \enquote{invisibility.}

\medskip
A mass-scale negative gesture against politics consists simply of not voting. \enquote{Apathy} (i.e. a healthy boredom with the weary Spectacle) keeps over half the nation from the polls; anarchism never accomplished as much! (Nor did anarchism have anything to do with the failure of the recent Census.) Again, there are positive parallels: \enquote{networking} as an alternative to politics is practiced at many levels of society, and non-hierarchic organization has attained popularity even outside the anarchist movement, simply because it works. (ACT UP and Earth First! are two examples. Alcoholics Anonymous, oddly enough, is another.)

\medskip
Refusal of Work can take the forms of absenteeism, on-job drunkenness, sabotage, and sheer inattention -- but it can also give rise to new modes of rebellion: more self- employment, participation in the \enquote{black} economy and \enquote{lavoro nero,} welfare scams and other criminal options, pot farming, etc. -- all more or less \enquote{invisible} activities compared to traditional leftist confrontational tactics such as the general strike.

\medskip
Refusal of the Church? Well, the \enquote{negative gesture} here probably consists of… watching television. But the positive alternatives include all sorts of non-authoritarian forms of spirituality, from \enquote{unchurched} Christianity to neo- paganism. The \enquote{Free Religions} as I like to call them --  small, self-created, half-serious/half-fun cults influenced by such currents as Discordianism and anarcho-Taoism -- are to be found all over marginal America, and provide a growing \enquote{fourth way} outside the mainstream churches, the televangelical bigots, and New Age vapidity and consumerism. It might also be said that the chief refusal of orthodoxy consists of the construction of \enquote{private moralities} in the Nietzschean sense: the spirituality of \enquote{free spirits.}

\medskip
The negative refusal of Home is \enquote{homelessness,} which most consider a form of victimization, not wishing to be forced into nomadology. But \enquote{homelessness} can in a sense be a virtue, an adventure -- so it appears, at least, to the huge international movement of the squatters, our modern hobos.

\medskip
The negative refusal of the Family is clearly divorce, or some other symptom of \enquote{breakdown.} The positive alternative springs from the realization that life can be happier without the nuclear family, whereupon a hundred flowers bloom -- from single parentage to group marriage to erotic affinity group. The \enquote{European Project} fights a major rearguard action in defense of \enquote{Family} -- oedipal misery lies at the heart of Control. Alternatives exist -- but they must remain in hiding, especially since the War against Sex of the 1980's and 1990's.

\medskip
What is the refusal of Art? The \enquote{negative gesture} is not to be found in the silly nihilism of an \enquote{Art Strike} or the defacing of some famous painting -- it is to be seen in the almost universal glassy-eyed boredom that creeps over most people at the very mention of the word. But what would the \enquote{positive gesture} consist of? Is it possible to imagine an aesthetics that does not engage, that removes itself from History and even from the Market? or at least tends to do so? which wants to replace representation with presence? How does presence make itself felt even in (or through) representation?

\medskip
\enquote{Chaos Linguistics} traces a presence which is continually disappearing from all orderings of language and meaning- systems; an elusive presence, evanescent, latif (\enquote{subtle,} a term in sufi alchemy) -- the Strange Attractor around which memes accrue, chaotically forming new and spontaneous orders. Here we have an aesthetics of the borderland between chaos and order, the margin, the area of \enquote{catastrophe} where the breakdown of the system can equal enlightenment. (Note: for an explanation of \enquote{Chaos Linguistics} see Appendix A, then please read this paragraph again.)

\medskip
The disappearance of the artist IS \enquote{the suppression and realization of art,} in Situationist terms. But from where do we vanish? And are we ever seen or heard of again? We go to Croatan -- what's our fate? All our art consists of a goodbye note to history -- \enquote{Gone To Croatan} -- but where is it, and what will we do there?

\medskip
First: We're not talking here about literally vanishing from the world and its future: -- no escape backward in time to paleolithic \enquote{original leisure society} -- no forever utopia, no backmountain hideaway, no island; also, no post- Revolutionary utopia -- most likely no Revolution at all! --  also, no VONU, no anarchist Space Stations -- nor do we accept a \enquote{Baudrillardian disappearance} into the silence of an ironic hyperconformity. I have no quarrel with any Rimbauds who escape Art for whatever Abyssinia they can find. But we can't build an aesthetics, even an aesthetics of disappearance, on the simple act of never coming back. By saying we're not an avant-garde and that there is no avant- garde, we've written our \enquote{Gone To Croatan} -- the question then becomes, how to envision \enquote{everyday life} in Croatan? particularly if we cannot say that Croatan exists in Time (Stone Age or Post-Revolution) or Space, either as utopia or as some forgotten midwestern town or as Abyssinia? Where and when is the world of unmediated creativity? If it can exist, it does exist -- but perhaps only as a sort of alternate reality which we so far have not learned to perceive. Where would we look for the seeds -- the weeds cracking through our sidewalks -- from this other world into our world? the clues, the right directions for searching? a finger pointing at the moon?

\medskip
I believe, or would at least like to propose, that the only solution to the \enquote{suppression and realization} of Art lies in the emergence of the TAZ. I would strongly reject the criticism that the TAZ itself is \enquote{nothing but} a work of art, although it may have some of the trappings. I do suggest that the TAZ is the only possible \enquote{time} and \enquote{place} for art to happen for the sheer pleasure of creative play, and as an actual contribution to the forces which allow the TAZ to cohere and manifest.

\medskip
Art in the World of Art has become a commodity; but deeper than that lies the problem of re-presentation itself, and the refusal of all mediation. In the TAZ art as a commodity will simply become impossible; it will instead be a condition of life. Mediation is harder to overcome, but the removal of all barriers between artists and \enquote{users} of art will tend toward a condition in which (as A.K. Coomaraswamy described it) \enquote{the artist is not a special sort of person, but every person is a special sort of artist.}

\medskip
In sum: disappearance is not necessarily a \enquote{catastrophe} --  except in the mathematical sense of \enquote{a sudden topological change.} All the positive gestures sketched here seem to involve various degrees of invisibility rather than traditional revolutionary confrontation. The \enquote{New Left} never really believed in its own existence till it saw itself on the Evening News. The New Autonomy, by contrast, will either infiltrate the media and subvert \enquote{it} from within -- or else never be \enquote{seen} at all. The TAZ exists not only beyond Control but also beyond definition, beyond gazing and naming as acts of enslaving, beyond the understanding of the State, beyond the State's ability to see.

%%%%%%%%%%%%%%%%%%%%%%%%%%%%%%%%%%%%%%%%%%%%%%
\section{Ratholes in the Babylon of Information}

\lettrine{T}{he} TAZ as a conscious radical tactic will emerge under certain conditions:

\medskip
Psychological liberation. That is, we must realize (make real) the moments and spaces in which freedom is not only possible but actual. We must know in what ways we are genuinely oppressed, and also in what ways we are self- repressed or ensnared in a fantasy in which ideas oppress us. WORK, for example, is a far more actual source of misery for most of us than legislative politics. Alienation is far more dangerous for us than toothless outdated dying ideologies. Mental addiction to \enquote{ideals} -- which in fact turn out to be mere projections of our resentment and sensations of victimization -- will never further our project. The TAZ is not a harbinger of some pie-in-the-sky Social Utopia to which we must sacrifice our lives that our children's children may breathe a bit of free air. The TAZ must be the scene of our present autonomy, but it can only exist on the condition that we already know ourselves as free beings.

\medskip
The counter-Net must expand. At present it reflects more abstraction than actuality. Zines and BBSs exchange information, which is part of the necessary groundwork of the TAZ, but very little of this information relates to concrete goods and services necessary for the autonomous life. We do not live in CyberSpace; to dream that we do is to fall into CyberGnosis, the false transcendence of the body. The TAZ is a physical place and we are either in it or not. All the senses must be involved. The Web is like a new sense in some ways, but it must be added to the others --  the others must not be subtracted from it, as in some horrible parody of the mystic trance. Without the Web, the full realization of the TAZ-complex would be impossible. But the Web is not the end in itself. It's a weapon.

\medskip
The apparatus of Control -- the \enquote{State} -- must (or so we must assume) continue to deliquesce and petrify simultaneously, must progress on its present course in which hysterical rigidity comes more and more to mask a vacuity, an abyss of power. As power \enquote{disappears,} our will to power must be disappearance. 

\medskip
We've already dealt with the question of whether the TAZ can be viewed \enquote{merely} as a work of art. But you will also demand to know whether it is more than a poor rat-hole in the Babylon of Information, or rather a maze of tunnels, more and more connected, but devoted only to the economic dead-end of piratical parasitism? I'll answer that I'd rather be a rat in the wall than a rat in the cage -- but I'll also insist that the TAZ transcends these categories.

\medskip
A world in which the TAZ succeeded in putting down roots might resemble the world envisioned by \enquote{P.M.} in his fantasy novel bolo'bolo. Perhaps the TAZ is a \enquote{proto-bolo.} But inasmuch as the TAZ exists now, it stands for much more than the mundanity of negativity or countercultural drop-out- ism. We've mentioned the festal aspect of the moment which is unControlled, and which adheres in spontaneous self- ordering, however brief. It is \enquote{epiphanic} -- a peak experience on the social as well as individual scale.

\medskip
Liberation is realized < em=\enquote{}> struggle -- this is the essence of Nietzsche's \enquote{self-overcoming.} The present thesis might also take for a sign Nietzsche's wandering. It is the precursor of the drift, in the Situ sense of the derive and Lyotard's definition of driftwork. We can foresee a whole new geography, a kind of pilgrimage-map in which holy sites are replaced by peak experiences and TAZs: a real science of psychotopography, perhaps to be called \enquote{geo-autonomy} or \enquote{anarchomancy.}.

\medskip
The TAZ involves a kind of ferality, a growth from tameness to wild(er)ness, a \enquote{return} which is also a step forward. It also demands a \enquote{yoga} of chaos, a project of \enquote{higher} orderings (of consciousness or simply of life) which are approached by \enquote{surfing the wave-front of chaos,} of complex dynamism. The TAZ is an art of life in continual rising up, wild but gentle -- a seducer not a rapist, a smuggler rather than a bloody pirate, a dancer not an eschatologist.

\medskip
Let us admit that we have attended parties where for one brief night a republic of gratified desires was attained. Shall we not confess that the politics of that night have more reality and force for us than those of, say, the entire U.S. Government? Some of the \enquote{parties} we've mentioned lasted for two or three years. Is this something worth imagining, worth fighting for? Let us study invisibility, webworking, psychic nomadism -- and who knows what we might attain?

        \begin{flushright}
        -- Spring Equinox, 1990
        \end{flushright}

%%%%%%%%%%%%%%%%%%%%%%%%%%%%%%%%%%%%%%%%%%%%%%
%%%%%%%%%%%%%%%%%%%%%%%%%%%%%%%%%%%%%%%%%%%%%%
\appendix
\section{Chaos Linguistics}

\lettrine{N}{ot} yet a science but a proposition: That certain problems in linguistics might be solved by viewing language as a complex dynamical system or \enquote{Chaos field.}

\medskip
Of all the responses to Saussure's linguistics, two have special interest here: the first, \enquote{antilinguistics,} can be traced -- in the modern period -- from Rimbaud's departure for Abyssinia; to Nietzsche's \enquote{I fear that while we still have grammar we have not yet killed God}; to dada; to Korzybski's \enquote{the Map is not the Territory}; to Burroughs' cut-ups and \enquote{breakthrough in the Gray Room}; to Zerzan's attack on language itself as representation and mediation.

\medskip
The second, Chomskyan Linguistics, with its belief in \enquote{universal grammar} and its tree diagrams, represents (I believe) an attempt to \enquote{save} language by discovering \enquote{hidden invariables,} much in the same way certain scientists are trying to \enquote{save} physics from the \enquote{irrationality} of quantum mechanics. Although as an anarchist Chomsky might have been expected to side with the nihilists, in fact his beautiful theory has more in common with platonism or sufism than with anarchism. Traditional metaphysics describes language as pure light shining through the colored glass of the archetypes; Chomsky speaks of \enquote{innate} grammars. Words are leaves, branches are sentences, mother tongues are limbs, language families are trunks, and the roots are in \enquote{heaven}… or the DNA. I call this \enquote{hermetalinguistics} -- hermetic and metaphysical. Nihilism (or \enquote{HeavyMetalinguistics} in honor of Burroughs) seems to me to have brought language to a dead end and threatened to render it \enquote{impossible} (a great feat, but a depressing one)- -while Chomsky holds out the promise and hope of a last- minute revelation, which I find equally difficult to accept. I too would like to \enquote{save} language, but without recourse to any \enquote{Spooks,} or supposed rules about God, dice, and the Universe.

\medskip
Returning to Saussure, and his posthumously published notes on anagrams in Latin poetry, we find certain hints of a process which somehow escapes the sign/signifier dynamic. Saussure was confronted with the suggestion of some sort of \enquote{meta}-linguistics which happens within language rather than being imposed as a categorical imperative from \enquote{outside.} As soon as language begins to play, as in the acrostic poems he examined, it seems to resonate with self- amplifying complexity. Saussure tried to quantify the anagrams but his figures kept running away from him (as if perhaps nonlinear equations were involved). Also, he began to find the anagrams everywhere, even in Latin prose. He began to wonder if he were hallucinating -- or if anagrams were a natural unconscious process of parole. He abandoned the project.

\medskip
I wonder: if enough of this sort of data were crunched through a computer, would we begin to be able to model language in terms of complex dynamical systems? Grammars then would not be \enquote{innate,} but would emerge from chaos as spontaneously evolving \enquote{higher orders,} in Prigogine's sense of \enquote{creative evolution.} Grammars could be thought of as \enquote{Strange Attractors,} like the hidden pattern which \enquote{caused} the anagrams -- patterns which are \enquote{real} but have \enquote{existence} only in terms of the sub-patterns they manifest. If meaning is elusive, perhaps it is because consciousness itself, and therefore language, is fractal.

\medskip
I find this theory more satisfyingly anarchistic than either anti-linguistics or Chomskyanism. It suggests that language can overcome representation and mediation, not because it is innate, but because it is chaos. It would suggest that all dadaistic experimentation (Feyerabend described his school of scientific epistemology as \enquote{anarchist dada}) in sound poetry, gesture, cut-up, beast languages, etc. -- all this was aimed neither at discovering nor destroying meaning, but at creating it. Nihilism points out gloomily that language \enquote{arbitrarily} creates meaning. Chaos Linguistics happily agrees, but adds that language can overcome language, that language can create freedom out of semantic tyranny's confusion and decay.

%%%%%%%%%%%%%%%%%%%%%%%%%%%%%%%%%%%%%%%%%%%%%%
\section{Applied Hedonics}

\lettrine{T}{he} bonnot gang were vegetarians and drank only water. They came to a bad (tho' picturesque) end. Vegetables and water, in themselves excellent things -- pure zen really -- shouldn't be consumed as martyrdom but as an epiphany. Self-denial as radical praxis, the Leveller impulse, tastes of millenarian gloom -- and this current on the Left shares an historical wellspring with the neo-puritan fundamentalism and moralic reaction of our decade. The New Ascesis, whether practiced by anorexic health-cranks, thin-lipped police sociologists, downtown straight-edge nihilists, cornpone fascist baptists, socialist torpedoes, drug-free Republicans… in every case the motive force is the same: resentment. resentment

\medskip
In the face of contemporary pecksniffian anaesthesia we'll erect a whole gallery of forebears, heros who carried on the struggle against bad consciousness but still knew how to party, a genial gene pool, a rare and difficult category to define, great minds not just for Truth but for the truth of pleasure, serious but not sober, whose sunny disposition makes them not sluggish but sharp, brilliant but not tormented. Imagine a Nietzsche with good digestion. Not the tepid Epicureans nor the bloated Sybarites. Sort of a spiritual hedonism, an actual Path of Pleasure, vision of a good life which is both noble and possible, rooted in a sense of the magnificent over-abundance of reality.

\medskip
\begin{center}
Shaykh Abu Sa'id of Khorassan\\
Charles Fourier\\
Brillat-Savarin\\
Rabelais\\
Abu Nuwas\\
Aga Khan III\\
R. Vaneigem\\
Oscar Wilde\\
Omar Khayyam\\
Sir Richard Burton\\
Emma Goldman\\

\medskip
add your own favorites
\end{center}

%%%%%%%%%%%%%%%%%%%%%%%%%%%%%%%%%%%%%%%%%%%%%%
\section{Extra Quotes}
%%%%%%%%%%%%%%%%%%%%%%%%%%%%%%%%%%%%%%%%%%%%%%
\subsection{Jalaloddin Rumi, Diwan-e Shams}

\begin{verse}
    As for us, He has appointed the job of permanent unemployment.\\
    If he wanted us to work, after all,\\
    He would not have created this wine. wine\\
    With a skinfull of this, Sir, this\\
    would you rush out to commit economics?\\
\end{verse}

\begin{flushright} -- Jalaloddin Rumi, Diwan-e Shams\end{flushright}

\subsection{Omar FitzGerald}

\begin{verse}
    Here with a Loaf of Bread beneath the Bough,\\
    A flask of Wine, A Book of Verse -- and Thou\\
    Beside me singing in the Wilderness -- \\
    And Wilderness is Paradise enow.\\
    Ah, my Beloved, fill the cup that clears\\
    To-day of past Regrets and future Fears -- \\
    Tomorrow? -- Why, Tomorrow I may be\\
    Myself with Yesterday's Sev'n Thousand Years.\\
    Ah, Love! could thou and I with Fate conspire\\
    To grasp this sorry Scheme of Things entire,\\
    Would not we shatter it to bits -- and then\\
    Re-mould it nearer to the Heart's Desire!\\
\end{verse}

\begin{flushright} -- Omar FitzGerald\end{flushright}

%%%%%%%%%%%%%%%%%%%%%%%%%%%%%%%%%%%%%%%%%%%%%%
\subsection{Renzo Novatore Arcola, January, 1920}

\lettrine{H}{istory}, materialism, monism, positivism, and all the \enquote{isms} of this world are old and rusty tools which I don't need or mind anymore. My principle is life, my end is death. I wish to live my life intensely for to embrace my life tragically.

\medskip
You are waiting for the revolution? My own began a long time ago! When you will be ready (God, what an endless wait!) I won't mind going along with you for awhile. But when you'll stop, I shall continue on my insane and triumphal way toward the great and sublime conquest of the nothing! Any society that you build will have its limits. And outside the limits of any society the unruly and heroic tramps will wander, with their wild \& virgin thoughts -- they who cannot live without planning ever new and dreadful outbursts of rebellion!

\medskip
\noindent I shall be among them!

\medskip
And after me, as before me, there will be those saying to their fellows: \enquote{So turn to yourselves rather than to your Gods or to your idols. Find what hides in yourselves; bring it to light; show yourselves!}

\medskip
Because every person; who, searching his own inwardness, extracts what was mysteriously hidden therein; is a shadow eclipsing any form of society which can exist under the sun! All societies tremble when the scornful aristocracy of the tramps, the inaccessibles, the uniques, the rulers over the ideal, and the conquerors of the nothing resolutely advances.

\medskip
\noindent So, come on iconoclasts, forward!

\medskip
\noindent \enquote{Already the foreboding sky grows dark and silent!}

\begin{flushright} -- Renzo Novatore Arcola, January, 1920\end{flushright}

%%%%%%%%%%%%%%%%%%%%%%%%%%%%%%%%%%%%%%%%%%%%%%
\subsection{Pirate Rant - Captain Bellamy}

\lettrine{D}{aniel Defoe}, writing under the pen name Captain Charles Johnson, wrote what became the first standard historical text on pirates, A General History of the Robberies and Murders of the Most Notorious Pirates. According to Patrick Pringle's Jolly Roger, pirate recruitment was most effective among the unemployed, escaped bondsmen, and transported criminals. The high seas made for an instantaneous levelling of class inequalities. Defoe relates that a pirate named Captain Bellamy made this speech to the captain of a merchant vessel he had taken as a prize. The captain of the merchant vessel had just declined an invitation to join the pirates.

\medskip
I am sorry they won't let you have your sloop again, for I scorn to do any one a mischief, when it is not to my advantage; damn the sloop, we must sink her, and she might be of use to you. Though you are a sneaking puppy, and so are all those who will submit to be governed by laws which rich men have made for their own security; for the cowardly whelps have not the courage otherwise to defend what they get by knavery; but damn ye altogether: damn them for a pack of crafty rascals, and you, who serve them, for a parcel of hen-hearted numbskulls. They vilify us, the scoundrels do, when there is only this difference, they rob the poor under the cover of law, forsooth, and we plunder the rich under the protection of our own courage. Had you not better make then one of us, than sneak after these villains for employment?

\medskip
When the captain replied that his conscience would not let him break the laws of God and man, the pirate Bellamy continued:

\medskip
You are a devilish conscience rascal, I am a free prince, and I have as much authority to make war on the whole world, as he who has a hundred sail of ships at sea, and an army of 100,000 men in the field; and this my conscience tells me: but there is no arguing with such snivelling puppies, who allow superiors to kick them about deck at pleasure.

%%%%%%%%%%%%%%%%%%%%%%%%%%%%%%%%%%%%%%%%%%%%%%
\subsection{The Dinner Party -- S. Pearl Andrews}

\lettrine{T}{he} highest type of human society in the existing social order is found in the parlor. In the elegant and refined reunions of the aristocratic classes there is none of the impertinent interference of legislation. The Individuality of each is fully admitted. Intercourse, therefore, is perfectly free. Conversation is continuous, brilliant, and varied. Groups are formed according to attraction. They are continuously broken up, and re-formed through the operation of the same subtile and all-pervading influence. Mutual deference pervades all classes, and the most perfect harmony, ever yet attained, in complex human relations, prevails under precisely those circumstances which Legislators and Statesmen dread as the conditions of inevitable anarchy and confusion. If there are laws of etiquette at all, they are mere suggestions of principles admitted into and judged of for himself or herself, by each individual mind.

\medskip
Is it conceivable that in all the future progress of humanity, with all the innumerable elements of development which the present age is unfolding, society generally, and in all its relations, will not attain as high a grade of perfection as certain portions of society, in certain special relations, have already attained?

\medskip
Suppose the intercourse of the parlor to be regulated by specific legislation. Let the time which each gentleman shall be allowed to speak to each lady be fixed by law; the position in which they should sit or stand be precisely regulated; the subjects which they shall be allowed to speak of, and the tone of voice and accompanying gestures with which each may be treated, carefully defined, all under pretext of preventing disorder and encroachment upon each other's privileges and rights, then can any thing be conceived better calculated or more certain to convert social intercourse into intolerable slavery and hopeless confusion?

\begin{flushright}-- S. Pearl Andrews The Science of Society\end{flushright}

\end{document}
